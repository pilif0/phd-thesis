\documentclass[class=smolathesis,crop=false]{standalone}

\begin{document}

\chapter{Appendix}
\label{ch:appendix}

\section{ILL Deduction Embedding Functions}
\label{app:ill-deduct}

The conclusions of deeply embedded ILL deductions (see Section~\ref{sec:linearity/deep}) are represented by values from the \isa{ill-sequent} datatype, capturing the antecedents and consequent:
\begin{isadef}[Datatype for arbitrary ILL sequents]{isa:ill_sequent}
  \isacomm{datatype}\isamarkupfalse%
\ \isatv{a}\ ill{\isacharunderscore}sequent\ {\isacharequal}\ Sequent \isapars{{\isachardoublequoteopen}\isatv{a}\ ill{\isacharunderscore}prop\ list{\isachardoublequoteclose}} \isapars{{\isachardoublequoteopen}\isatv{a}\ ill{\isacharunderscore}prop{\isachardoublequoteclose}}

\end{isadef}

We tie those values to the shallowly embedded relation of valid sequents (see Section~\ref{sec:linearity/shallow}) through the \isa{ill-sequent-valid} function:
\begin{isadef}[Validity predicate on ILL sequents]{isa:ill_sequent_valid}
  \isacomm{primrec}\isamarkupfalse%
\ ill{\isacharunderscore}sequent{\isacharunderscore}valid\ {\isacharcolon}{\isacharcolon}\ {\isachardoublequoteopen}\isatv{a}\ ill{\isacharunderscore}sequent\ {\isasymRightarrow}\ bool{\isachardoublequoteclose}\isanewline
\ \ \isaOcomm{where}\ {\isachardoublequoteopen}\isafv{ill{\isacharunderscore}sequent{\isacharunderscore}valid}\ {\isacharparenleft}Sequent\ \isabv{a\ c}{\isacharparenright}\ {\isacharequal}\ \isabv{a}\ {\isasymturnstile}\ \isabv{c}{\isachardoublequoteclose}

\end{isadef}

We set up a declaration bundle \isa{deep-sequent} which we can use to switch the turnstile notation from the relation to these values:
\begin{isadef}[Declaration bundle switching turnstile notation]{isa:deep_sequent}
  \isacomm{bundle}\isamarkupfalse%
\ deep{\isacharunderscore}sequent\ %
\isaOcomm{begin}\isanewline
\isanewline
\isacomm{no{\isacharunderscore}notation}\isamarkupfalse%
\ sequent\ {\isacharparenleft}\isaOcomm{infix}\ {\isachardoublequoteopen}{\isasymturnstile}{\isachardoublequoteclose}\ {\isadigit{6}}{\isadigit{0}}{\isacharparenright}\isanewline
\isacomm{notation}\isamarkupfalse%
\ Sequent\ {\isacharparenleft}\isaOcomm{infix}\ {\isachardoublequoteopen}{\isasymturnstile}{\isachardoublequoteclose}\ {\isadigit{6}}{\isadigit{0}}{\isacharparenright}\isanewline
\isanewline
\isaOcomm{end}\isamarkupfalse%

\end{isadef}

The full definition for conclusions of deeply embedded ILL deductions is then as follows:
\begin{isadef}[Conclusions of deeply embedded ILL deductions]{isa:ill_conclusion}
  \isacomm{primrec}\ ill{\isacharunderscore}conclusion\ {\isacharcolon}{\isacharcolon}\ {\isachardoublequoteopen}{\isacharparenleft}\isatv{a}{\isacharcomma}\ \isatv{l}{\isacharparenright}\ ill{\isacharunderscore}deduct\ {\isasymRightarrow}\ \isatv{a}\ ill{\isacharunderscore}sequent{\isachardoublequoteclose}\ \isaOcomm{where}\isanewline
\ \ \ \ {\isachardoublequoteopen}\isafv{ill{\isacharunderscore}conclusion}\ {\isacharparenleft}Premise\ \isabv{G}\ \isabv{c}\ \isabv{l}{\isacharparenright}\ {\isacharequal}\ \isabv{G}\ {\isasymturnstile}\ \isabv{c}{\isachardoublequoteclose}\isanewline
\ \ {\isacharbar}\ {\isachardoublequoteopen}\isafv{ill{\isacharunderscore}conclusion}\ {\isacharparenleft}Identity\ \isabv{a}{\isacharparenright}\ {\isacharequal}\ {\isacharbrackleft}\isabv{a}{\isacharbrackright}\ {\isasymturnstile}\ \isabv{a}{\isachardoublequoteclose}\isanewline
\ \ {\isacharbar}\ {\isachardoublequoteopen}\isafv{ill{\isacharunderscore}conclusion}\ {\isacharparenleft}Exchange\ \isabv{G}\ \isabv{a}\ \isabv{b}\ \isabv{D}\ \isabv{c}\ \isabv{P}{\isacharparenright}\ {\isacharequal}\ \isabv{G}\ {\isacharat}\ {\isacharbrackleft}\isabv{b}{\isacharbrackright}\ {\isacharat}\ {\isacharbrackleft}\isabv{a}{\isacharbrackright}\ {\isacharat}\ \isabv{D}\ {\isasymturnstile}\ \isabv{c}{\isachardoublequoteclose}\isanewline
\ \ {\isacharbar}\ {\isachardoublequoteopen}\isafv{ill{\isacharunderscore}conclusion}\ {\isacharparenleft}Cut\ \isabv{G}\ \isabv{b}\ \isabv{D}\ \isabv{E}\ \isabv{c}\ \isabv{P}\ \isabv{Q}{\isacharparenright}\ {\isacharequal}\ \isabv{D}\ {\isacharat}\ \isabv{G}\ {\isacharat}\ \isabv{E}\ {\isasymturnstile}\ \isabv{c}{\isachardoublequoteclose}\isanewline
\ \ {\isacharbar}\ {\isachardoublequoteopen}\isafv{ill{\isacharunderscore}conclusion}\ {\isacharparenleft}TimesL\ \isabv{G}\ \isabv{a}\ \isabv{b}\ \isabv{D}\ \isabv{c}\ \isabv{P}{\isacharparenright}\ {\isacharequal}\ \isabv{G}\ {\isacharat}\ {\isacharbrackleft}\isabv{a}\ {\isasymotimes}\ \isabv{b}{\isacharbrackright}\ {\isacharat}\ \isabv{D}\ {\isasymturnstile}\ \isabv{c}{\isachardoublequoteclose}\isanewline
\ \ {\isacharbar}\ {\isachardoublequoteopen}\isafv{ill{\isacharunderscore}conclusion}\ {\isacharparenleft}TimesR\ \isabv{G}\ \isabv{a}\ \isabv{D}\ \isabv{b}\ \isabv{P}\ \isabv{Q}{\isacharparenright}\ {\isacharequal}\ \isabv{G}\ {\isacharat}\ \isabv{D}\ {\isasymturnstile}\ \isabv{a}\ {\isasymotimes}\ \isabv{b}{\isachardoublequoteclose}\isanewline
\ \ {\isacharbar}\ {\isachardoublequoteopen}\isafv{ill{\isacharunderscore}conclusion}\ {\isacharparenleft}OneL\ \isabv{G}\ \isabv{D}\ \isabv{c}\ \isabv{P}{\isacharparenright}\ {\isacharequal}\ \isabv{G}\ {\isacharat}\ {\isacharbrackleft}{\isasymone}{\isacharbrackright}\ {\isacharat}\ \isabv{D}\ {\isasymturnstile}\ \isabv{c}{\isachardoublequoteclose}\isanewline
\ \ {\isacharbar}\ {\isachardoublequoteopen}\isafv{ill{\isacharunderscore}conclusion}\ {\isacharparenleft}OneR{\isacharparenright}\ {\isacharequal}\ {\isacharbrackleft}{\isacharbrackright}\ {\isasymturnstile}\ {\isasymone}{\isachardoublequoteclose}\isanewline
\ \ {\isacharbar}\ {\isachardoublequoteopen}\isafv{ill{\isacharunderscore}conclusion}\ {\isacharparenleft}LimpL\ \isabv{G}\ \isabv{a}\ \isabv{D}\ \isabv{b}\ \isabv{E}\ \isabv{c}\ \isabv{P}\ \isabv{Q}{\isacharparenright}\ {\isacharequal}\ \isabv{G}\ {\isacharat}\ \isabv{D}\ {\isacharat}\ {\isacharbrackleft}\isabv{a}\ {\isasymrhd}\ \isabv{b}{\isacharbrackright}\ {\isacharat}\ \isabv{E}\ {\isasymturnstile}\ \isabv{c}{\isachardoublequoteclose}\isanewline
\ \ {\isacharbar}\ {\isachardoublequoteopen}\isafv{ill{\isacharunderscore}conclusion}\ {\isacharparenleft}LimpR\ \isabv{G}\ \isabv{a}\ \isabv{D}\ \isabv{b}\ \isabv{P}{\isacharparenright}\ {\isacharequal}\ \isabv{G}\ {\isacharat}\ \isabv{D}\ {\isasymturnstile}\ \isabv{a}\ {\isasymrhd}\ \isabv{b}{\isachardoublequoteclose}\isanewline
\ \ {\isacharbar}\ {\isachardoublequoteopen}\isafv{ill{\isacharunderscore}conclusion}\ {\isacharparenleft}WithL{\isadigit{1}}\ \isabv{G}\ \isabv{a}\ \isabv{b}\ \isabv{D}\ \isabv{c}\ \isabv{P}{\isacharparenright}\ {\isacharequal}\ \isabv{G}\ {\isacharat}\ {\isacharbrackleft}\isabv{a}\ {\isacharampersand}\ \isabv{b}{\isacharbrackright}\ {\isacharat}\ \isabv{D}\ {\isasymturnstile}\ \isabv{c}{\isachardoublequoteclose}\isanewline
\ \ {\isacharbar}\ {\isachardoublequoteopen}\isafv{ill{\isacharunderscore}conclusion}\ {\isacharparenleft}WithL{\isadigit{2}}\ \isabv{G}\ \isabv{a}\ \isabv{b}\ \isabv{D}\ \isabv{c}\ \isabv{P}{\isacharparenright}\ {\isacharequal}\ \isabv{G}\ {\isacharat}\ {\isacharbrackleft}\isabv{a}\ {\isacharampersand}\ \isabv{b}{\isacharbrackright}\ {\isacharat}\ \isabv{D}\ {\isasymturnstile}\ \isabv{c}{\isachardoublequoteclose}\isanewline
\ \ {\isacharbar}\ {\isachardoublequoteopen}\isafv{ill{\isacharunderscore}conclusion}\ {\isacharparenleft}WithR\ \isabv{G}\ \isabv{a}\ \isabv{b}\ \isabv{P}\ \isabv{Q}{\isacharparenright}\ {\isacharequal}\ \isabv{G}\ {\isasymturnstile}\ \isabv{a}\ {\isacharampersand}\ \isabv{b}{\isachardoublequoteclose}\isanewline
\ \ {\isacharbar}\ {\isachardoublequoteopen}\isafv{ill{\isacharunderscore}conclusion}\ {\isacharparenleft}TopR\ \isabv{G}{\isacharparenright}\ {\isacharequal}\ \isabv{G}\ {\isasymturnstile}\ {\isasymtop}{\isachardoublequoteclose}\isanewline
\ \ {\isacharbar}\ {\isachardoublequoteopen}\isafv{ill{\isacharunderscore}conclusion}\ {\isacharparenleft}PlusL\ \isabv{G}\ \isabv{a}\ \isabv{b}\ \isabv{D}\ \isabv{c}\ \isabv{P}\ \isabv{Q}{\isacharparenright}\ {\isacharequal}\ \isabv{G}\ {\isacharat}\ {\isacharbrackleft}\isabv{a}\ {\isasymoplus}\ \isabv{b}{\isacharbrackright}\ {\isacharat}\ \isabv{D}\ {\isasymturnstile}\ \isabv{c}{\isachardoublequoteclose}\isanewline
\ \ {\isacharbar}\ {\isachardoublequoteopen}\isafv{ill{\isacharunderscore}conclusion}\ {\isacharparenleft}PlusR{\isadigit{1}}\ \isabv{G}\ \isabv{a}\ \isabv{b}\ \isabv{P}{\isacharparenright}\ {\isacharequal}\ \isabv{G}\ {\isasymturnstile}\ \isabv{a}\ {\isasymoplus}\ \isabv{b}{\isachardoublequoteclose}\isanewline
\ \ {\isacharbar}\ {\isachardoublequoteopen}\isafv{ill{\isacharunderscore}conclusion}\ {\isacharparenleft}PlusR{\isadigit{2}}\ \isabv{G}\ \isabv{a}\ \isabv{b}\ \isabv{P}{\isacharparenright}\ {\isacharequal}\ \isabv{G}\ {\isasymturnstile}\ \isabv{a}\ {\isasymoplus}\ \isabv{b}{\isachardoublequoteclose}\isanewline
\ \ {\isacharbar}\ {\isachardoublequoteopen}\isafv{ill{\isacharunderscore}conclusion}\ {\isacharparenleft}ZeroL\ \isabv{G}\ \isabv{D}\ \isabv{c}{\isacharparenright}\ {\isacharequal}\ \isabv{G}\ {\isacharat}\ {\isacharbrackleft}{\isasymzero}{\isacharbrackright}\ {\isacharat}\ \isabv{D}\ {\isasymturnstile}\ \isabv{c}{\isachardoublequoteclose}\isanewline
\ \ {\isacharbar}\ {\isachardoublequoteopen}\isafv{ill{\isacharunderscore}conclusion}\ {\isacharparenleft}Weaken\ \isabv{G}\ \isabv{D}\ \isabv{b}\ \isabv{a}\ \isabv{P}{\isacharparenright}\ {\isacharequal}\ \isabv{G}\ {\isacharat}\ {\isacharbrackleft}{\isacharbang}\isabv{a}{\isacharbrackright}\ {\isacharat}\ \isabv{D}\ {\isasymturnstile}\ \isabv{b}{\isachardoublequoteclose}\isanewline
\ \ {\isacharbar}\ {\isachardoublequoteopen}\isafv{ill{\isacharunderscore}conclusion}\ {\isacharparenleft}Contract\ \isabv{G}\ \isabv{a}\ \isabv{D}\ \isabv{b}\ \isabv{P}{\isacharparenright}\ {\isacharequal}\ \isabv{G}\ {\isacharat}\ {\isacharbrackleft}{\isacharbang}\isabv{a}{\isacharbrackright}\ {\isacharat}\ \isabv{D}\ {\isasymturnstile}\ \isabv{b}{\isachardoublequoteclose}\isanewline
\ \ {\isacharbar}\ {\isachardoublequoteopen}\isafv{ill{\isacharunderscore}conclusion}\ {\isacharparenleft}Derelict\ \isabv{G}\ \isabv{a}\ \isabv{D}\ \isabv{b}\ \isabv{P}{\isacharparenright}\ {\isacharequal}\ \isabv{G}\ {\isacharat}\ {\isacharbrackleft}{\isacharbang}\isabv{a}{\isacharbrackright}\ {\isacharat}\ \isabv{D}\ {\isasymturnstile}\ \isabv{b}{\isachardoublequoteclose}\isanewline
\ \ {\isacharbar}\ {\isachardoublequoteopen}\isafv{ill{\isacharunderscore}conclusion}\ {\isacharparenleft}Promote\ \isabv{G}\ \isabv{a}\ \isabv{P}{\isacharparenright}\ {\isacharequal}\ map\ Exp\ \isabv{G}\ {\isasymturnstile}\ {\isacharbang}\isabv{a}{\isachardoublequoteclose}

\end{isadef}

Their well-formedness is defined using these conclusions as follows:
\begin{isadef}[Well-formedness of deeply embedded ILL deductions]{isa:ill_deduct_wf}
  \isacomm{primrec}\isamarkupfalse%
\ ill{\isacharunderscore}deduct{\isacharunderscore}wf\ {\isacharcolon}{\isacharcolon}\ {\isachardoublequoteopen}{\isacharparenleft}\isatv{a}{\isacharcomma}\ \isatv{l}{\isacharparenright}\ ill{\isacharunderscore}deduct\ {\isasymRightarrow}\ bool{\isachardoublequoteclose}\ \isaOcomm{where}\isanewline
\ \ \ \ {\isachardoublequoteopen}\isafv{ill{\isacharunderscore}deduct{\isacharunderscore}wf}\ {\isacharparenleft}Premise\ \isabv{G}\ \isabv{c}\ \isabv{l}{\isacharparenright}\ {\isacharequal}\ True{\isachardoublequoteclose}\isanewline
\ \ {\isacharbar}\ {\isachardoublequoteopen}\isafv{ill{\isacharunderscore}deduct{\isacharunderscore}wf}\ {\isacharparenleft}Identity\ \isabv{a}{\isacharparenright}\ {\isacharequal}\ True{\isachardoublequoteclose}\isanewline
\ \ {\isacharbar}\ {\isachardoublequoteopen}\isafv{ill{\isacharunderscore}deduct{\isacharunderscore}wf}\ {\isacharparenleft}Exchange\ \isabv{G}\ \isabv{a}\ \isabv{b}\ \isabv{D}\ \isabv{c}\ \isabv{P}{\isacharparenright}\ {\isacharequal}\isanewline
\ \ \ \ \ \ {\isacharparenleft}\isafv{ill{\isacharunderscore}deduct{\isacharunderscore}wf}\ \isabv{P}\ {\isasymand}\ ill{\isacharunderscore}conclusion\ \isabv{P}\ {\isacharequal}\ \isabv{G}\ {\isacharat}\ {\isacharbrackleft}\isabv{a}{\isacharbrackright}\ {\isacharat}\ {\isacharbrackleft}\isabv{b}{\isacharbrackright}\ {\isacharat}\ \isabv{D}\ {\isasymturnstile}\ \isabv{c}{\isacharparenright}{\isachardoublequoteclose}\isanewline
\ \ {\isacharbar}\ {\isachardoublequoteopen}\isafv{ill{\isacharunderscore}deduct{\isacharunderscore}wf}\ {\isacharparenleft}Cut\ \isabv{G}\ \isabv{b}\ \isabv{D}\ \isabv{E}\ \isabv{c}\ \isabv{P}\ \isabv{Q}{\isacharparenright}\ {\isacharequal}\isanewline
\ \ \ \ \ \ {\isacharparenleft}\ \isafv{ill{\isacharunderscore}deduct{\isacharunderscore}wf}\ \isabv{P}\ {\isasymand}\ ill{\isacharunderscore}conclusion\ \isabv{P}\ {\isacharequal}\ \isabv{G}\ {\isasymturnstile}\ \isabv{b}\ {\isasymand}\isanewline
\ \ \ \ \ \ \ \ \isafv{ill{\isacharunderscore}deduct{\isacharunderscore}wf}\ \isabv{Q}\ {\isasymand}\ ill{\isacharunderscore}conclusion\ \isabv{Q}\ {\isacharequal}\ \isabv{D}\ {\isacharat}\ {\isacharbrackleft}\isabv{b}{\isacharbrackright}\ {\isacharat}\ \isabv{E}\ {\isasymturnstile}\ \isabv{c}{\isacharparenright}{\isachardoublequoteclose}\isanewline
\ \ {\isacharbar}\ {\isachardoublequoteopen}\isafv{ill{\isacharunderscore}deduct{\isacharunderscore}wf}\ {\isacharparenleft}TimesL\ \isabv{G}\ \isabv{a}\ \isabv{b}\ \isabv{D}\ \isabv{c}\ \isabv{P}{\isacharparenright}\ {\isacharequal}\isanewline
\ \ \ \ \ \ {\isacharparenleft}\isafv{ill{\isacharunderscore}deduct{\isacharunderscore}wf}\ \isabv{P}\ {\isasymand}\ ill{\isacharunderscore}conclusion\ \isabv{P}\ {\isacharequal}\ \isabv{G}\ {\isacharat}\ {\isacharbrackleft}\isabv{a}{\isacharbrackright}\ {\isacharat}\ {\isacharbrackleft}\isabv{b}{\isacharbrackright}\ {\isacharat}\ \isabv{D}\ {\isasymturnstile}\ \isabv{c}{\isacharparenright}{\isachardoublequoteclose}\isanewline
\ \ {\isacharbar}\ {\isachardoublequoteopen}\isafv{ill{\isacharunderscore}deduct{\isacharunderscore}wf}\ {\isacharparenleft}TimesR\ \isabv{G}\ \isabv{a}\ \isabv{D}\ \isabv{b}\ \isabv{P}\ \isabv{Q}{\isacharparenright}\ {\isacharequal}\isanewline
\ \ \ \ \ \ {\isacharparenleft}\ \isafv{ill{\isacharunderscore}deduct{\isacharunderscore}wf}\ \isabv{P}\ {\isasymand}\ ill{\isacharunderscore}conclusion\ \isabv{P}\ {\isacharequal}\ \isabv{G}\ {\isasymturnstile}\ \isabv{a}\ {\isasymand}\isanewline
\ \ \ \ \ \ \ \ \isafv{ill{\isacharunderscore}deduct{\isacharunderscore}wf}\ \isabv{Q}\ {\isasymand}\ ill{\isacharunderscore}conclusion\ \isabv{Q}\ {\isacharequal}\ \isabv{D}\ {\isasymturnstile}\ \isabv{b}{\isacharparenright}{\isachardoublequoteclose}\isanewline
\ \ {\isacharbar}\ {\isachardoublequoteopen}\isafv{ill{\isacharunderscore}deduct{\isacharunderscore}wf}\ {\isacharparenleft}OneL\ \isabv{G}\ \isabv{D}\ \isabv{c}\ \isabv{P}{\isacharparenright}\ {\isacharequal}\isanewline
\ \ \ \ \ \ {\isacharparenleft}\isafv{ill{\isacharunderscore}deduct{\isacharunderscore}wf}\ \isabv{P}\ {\isasymand}\ ill{\isacharunderscore}conclusion\ \isabv{P}\ {\isacharequal}\ \isabv{G}\ {\isacharat}\ \isabv{D}\ {\isasymturnstile}\ \isabv{c}{\isacharparenright}{\isachardoublequoteclose}\isanewline
\ \ {\isacharbar}\ {\isachardoublequoteopen}\isafv{ill{\isacharunderscore}deduct{\isacharunderscore}wf}\ {\isacharparenleft}OneR{\isacharparenright}\ {\isacharequal}\ True{\isachardoublequoteclose}\isanewline
\ \ {\isacharbar}\ {\isachardoublequoteopen}\isafv{ill{\isacharunderscore}deduct{\isacharunderscore}wf}\ {\isacharparenleft}LimpL\ \isabv{G}\ \isabv{a}\ \isabv{D}\ \isabv{b}\ \isabv{E}\ \isabv{c}\ \isabv{P}\ \isabv{Q}{\isacharparenright}\ {\isacharequal}\isanewline
\ \ \ \ \ \ {\isacharparenleft}\ \isafv{ill{\isacharunderscore}deduct{\isacharunderscore}wf}\ \isabv{P}\ {\isasymand}\ ill{\isacharunderscore}conclusion\ \isabv{P}\ {\isacharequal}\ \isabv{G}\ {\isasymturnstile}\ \isabv{a}\ {\isasymand}\isanewline
\ \ \ \ \ \ \ \ \isafv{ill{\isacharunderscore}deduct{\isacharunderscore}wf}\ \isabv{Q}\ {\isasymand}\ ill{\isacharunderscore}conclusion\ \isabv{Q}\ {\isacharequal}\ \isabv{D}\ {\isacharat}\ {\isacharbrackleft}\isabv{b}{\isacharbrackright}\ {\isacharat}\ \isabv{E}\ {\isasymturnstile}\ \isabv{c}{\isacharparenright}{\isachardoublequoteclose}\isanewline
\ \ {\isacharbar}\ {\isachardoublequoteopen}\isafv{ill{\isacharunderscore}deduct{\isacharunderscore}wf}\ {\isacharparenleft}LimpR\ \isabv{G}\ \isabv{a}\ \isabv{D}\ \isabv{b}\ \isabv{P}{\isacharparenright}\ {\isacharequal}\isanewline
\ \ \ \ \ \ {\isacharparenleft}\isafv{ill{\isacharunderscore}deduct{\isacharunderscore}wf}\ \isabv{P}\ {\isasymand}\ ill{\isacharunderscore}conclusion\ \isabv{P}\ {\isacharequal}\ \isabv{G}\ {\isacharat}\ {\isacharbrackleft}\isabv{a}{\isacharbrackright}\ {\isacharat}\ \isabv{D}\ {\isasymturnstile}\ \isabv{b}{\isacharparenright}{\isachardoublequoteclose}\isanewline
\ \ {\isacharbar}\ {\isachardoublequoteopen}\isafv{ill{\isacharunderscore}deduct{\isacharunderscore}wf}\ {\isacharparenleft}WithL{\isadigit{1}}\ \isabv{G}\ \isabv{a}\ \isabv{b}\ \isabv{D}\ \isabv{c}\ \isabv{P}{\isacharparenright}\ {\isacharequal}\isanewline
\ \ \ \ \ \ {\isacharparenleft}\isafv{ill{\isacharunderscore}deduct{\isacharunderscore}wf}\ \isabv{P}\ {\isasymand}\ ill{\isacharunderscore}conclusion\ \isabv{P}\ {\isacharequal}\ \isabv{G}\ {\isacharat}\ {\isacharbrackleft}\isabv{a}{\isacharbrackright}\ {\isacharat}\ \isabv{D}\ {\isasymturnstile}\ \isabv{c}{\isacharparenright}{\isachardoublequoteclose}\isanewline
\ \ {\isacharbar}\ {\isachardoublequoteopen}\isafv{ill{\isacharunderscore}deduct{\isacharunderscore}wf}\ {\isacharparenleft}WithL{\isadigit{2}}\ \isabv{G}\ \isabv{a}\ \isabv{b}\ \isabv{D}\ \isabv{c}\ \isabv{P}{\isacharparenright}\ {\isacharequal}\isanewline
\ \ \ \ \ \ {\isacharparenleft}\isafv{ill{\isacharunderscore}deduct{\isacharunderscore}wf}\ \isabv{P}\ {\isasymand}\ ill{\isacharunderscore}conclusion\ \isabv{P}\ {\isacharequal}\ \isabv{G}\ {\isacharat}\ {\isacharbrackleft}\isabv{b}{\isacharbrackright}\ {\isacharat}\ \isabv{D}\ {\isasymturnstile}\ \isabv{c}{\isacharparenright}{\isachardoublequoteclose}\isanewline
\ \ {\isacharbar}\ {\isachardoublequoteopen}\isafv{ill{\isacharunderscore}deduct{\isacharunderscore}wf}\ {\isacharparenleft}WithR\ \isabv{G}\ \isabv{a}\ \isabv{b}\ \isabv{P}\ \isabv{Q}{\isacharparenright}\ {\isacharequal}\isanewline
\ \ \ \ \ \ {\isacharparenleft}\ \isafv{ill{\isacharunderscore}deduct{\isacharunderscore}wf}\ \isabv{P}\ {\isasymand}\ ill{\isacharunderscore}conclusion\ \isabv{P}\ {\isacharequal}\ \isabv{G}\ {\isasymturnstile}\ \isabv{a}\ {\isasymand}\isanewline
\ \ \ \ \ \ \ \ \isafv{ill{\isacharunderscore}deduct{\isacharunderscore}wf}\ \isabv{Q}\ {\isasymand}\ ill{\isacharunderscore}conclusion\ \isabv{Q}\ {\isacharequal}\ \isabv{G}\ {\isasymturnstile}\ \isabv{b}{\isacharparenright}{\isachardoublequoteclose}\isanewline
\ \ {\isacharbar}\ {\isachardoublequoteopen}\isafv{ill{\isacharunderscore}deduct{\isacharunderscore}wf}\ {\isacharparenleft}TopR\ \isabv{G}{\isacharparenright}\ {\isacharequal}\ True{\isachardoublequoteclose}\isanewline
\ \ {\isacharbar}\ {\isachardoublequoteopen}\isafv{ill{\isacharunderscore}deduct{\isacharunderscore}wf}\ {\isacharparenleft}PlusL\ \isabv{G}\ \isabv{a}\ \isabv{b}\ \isabv{D}\ \isabv{c}\ \isabv{P}\ \isabv{Q}{\isacharparenright}\ {\isacharequal}\isanewline
\ \ \ \ \ \ {\isacharparenleft}\ \isafv{ill{\isacharunderscore}deduct{\isacharunderscore}wf}\ \isabv{P}\ {\isasymand}\ ill{\isacharunderscore}conclusion\ \isabv{P}\ {\isacharequal}\ \isabv{G}\ {\isacharat}\ {\isacharbrackleft}\isabv{a}{\isacharbrackright}\ {\isacharat}\ \isabv{D}\ {\isasymturnstile}\ \isabv{c}\ {\isasymand}\isanewline
\ \ \ \ \ \ \ \ \isafv{ill{\isacharunderscore}deduct{\isacharunderscore}wf}\ \isabv{Q}\ {\isasymand}\ ill{\isacharunderscore}conclusion\ \isabv{Q}\ {\isacharequal}\ \isabv{G}\ {\isacharat}\ {\isacharbrackleft}\isabv{b}{\isacharbrackright}\ {\isacharat}\ \isabv{D}\ {\isasymturnstile}\ \isabv{c}{\isacharparenright}{\isachardoublequoteclose}\isanewline
\ \ {\isacharbar}\ {\isachardoublequoteopen}\isafv{ill{\isacharunderscore}deduct{\isacharunderscore}wf}\ {\isacharparenleft}PlusR{\isadigit{1}}\ \isabv{G}\ \isabv{a}\ \isabv{b}\ \isabv{P}{\isacharparenright}\ {\isacharequal}\isanewline
\ \ \ \ \ \ {\isacharparenleft}\isafv{ill{\isacharunderscore}deduct{\isacharunderscore}wf}\ \isabv{P}\ {\isasymand}\ ill{\isacharunderscore}conclusion\ \isabv{P}\ {\isacharequal}\ \isabv{G}\ {\isasymturnstile}\ \isabv{a}{\isacharparenright}{\isachardoublequoteclose}\isanewline
\ \ {\isacharbar}\ {\isachardoublequoteopen}\isafv{ill{\isacharunderscore}deduct{\isacharunderscore}wf}\ {\isacharparenleft}PlusR{\isadigit{2}}\ \isabv{G}\ \isabv{a}\ \isabv{b}\ \isabv{P}{\isacharparenright}\ {\isacharequal}\isanewline
\ \ \ \ \ \ {\isacharparenleft}\isafv{ill{\isacharunderscore}deduct{\isacharunderscore}wf}\ \isabv{P}\ {\isasymand}\ ill{\isacharunderscore}conclusion\ \isabv{P}\ {\isacharequal}\ \isabv{G}\ {\isasymturnstile}\ \isabv{b}{\isacharparenright}{\isachardoublequoteclose}\isanewline
\ \ {\isacharbar}\ {\isachardoublequoteopen}\isafv{ill{\isacharunderscore}deduct{\isacharunderscore}wf}\ {\isacharparenleft}ZeroL\ \isabv{G}\ \isabv{D}\ \isabv{c}{\isacharparenright}\ {\isacharequal}\ True{\isachardoublequoteclose}\isanewline
\ \ {\isacharbar}\ {\isachardoublequoteopen}\isafv{ill{\isacharunderscore}deduct{\isacharunderscore}wf}\ {\isacharparenleft}Weaken\ \isabv{G}\ \isabv{D}\ \isabv{b}\ \isabv{a}\ \isabv{P}{\isacharparenright}\ {\isacharequal}\isanewline
\ \ \ \ \ \ {\isacharparenleft}\isafv{ill{\isacharunderscore}deduct{\isacharunderscore}wf}\ \isabv{P}\ {\isasymand}\ ill{\isacharunderscore}conclusion\ \isabv{P}\ {\isacharequal}\ \isabv{G}\ {\isacharat}\ \isabv{D}\ {\isasymturnstile}\ \isabv{b}{\isacharparenright}{\isachardoublequoteclose}\isanewline
\ \ {\isacharbar}\ {\isachardoublequoteopen}\isafv{ill{\isacharunderscore}deduct{\isacharunderscore}wf}\ {\isacharparenleft}Contract\ \isabv{G}\ \isabv{a}\ \isabv{D}\ \isabv{b}\ \isabv{P}{\isacharparenright}\ {\isacharequal}\isanewline
\ \ \ \ \ \ {\isacharparenleft}\isafv{ill{\isacharunderscore}deduct{\isacharunderscore}wf}\ \isabv{P}\ {\isasymand}\ ill{\isacharunderscore}conclusion\ \isabv{P}\ {\isacharequal}\ \isabv{G}\ {\isacharat}\ {\isacharbrackleft}{\isacharbang}\isabv{a}{\isacharbrackright}\ {\isacharat}\ {\isacharbrackleft}{\isacharbang}\isabv{a}{\isacharbrackright}\ {\isacharat}\ \isabv{D}\ {\isasymturnstile}\ \isabv{b}{\isacharparenright}{\isachardoublequoteclose}\isanewline
\ \ {\isacharbar}\ {\isachardoublequoteopen}\isafv{ill{\isacharunderscore}deduct{\isacharunderscore}wf}\ {\isacharparenleft}Derelict\ \isabv{G}\ \isabv{a}\ \isabv{D}\ \isabv{b}\ \isabv{P}{\isacharparenright}\ {\isacharequal}\isanewline
\ \ \ \ \ \ {\isacharparenleft}\isafv{ill{\isacharunderscore}deduct{\isacharunderscore}wf}\ \isabv{P}\ {\isasymand}\ ill{\isacharunderscore}conclusion\ \isabv{P}\ {\isacharequal}\ \isabv{G}\ {\isacharat}\ {\isacharbrackleft}\isabv{a}{\isacharbrackright}\ {\isacharat}\ \isabv{D}\ {\isasymturnstile}\ \isabv{b}{\isacharparenright}{\isachardoublequoteclose}\isanewline
\ \ {\isacharbar}\ {\isachardoublequoteopen}\isafv{ill{\isacharunderscore}deduct{\isacharunderscore}wf}\ {\isacharparenleft}Promote\ \isabv{G}\ \isabv{a}\ \isabv{P}{\isacharparenright}\ {\isacharequal}\isanewline
\ \ \ \ \ \ {\isacharparenleft}\isafv{ill{\isacharunderscore}deduct{\isacharunderscore}wf}\ \isabv{P}\ {\isasymand}\ ill{\isacharunderscore}conclusion\ \isabv{P}\ {\isacharequal}\ map\ Exp\ \isabv{G}\ {\isasymturnstile}\ \isabv{a}{\isacharparenright}{\isachardoublequoteclose}

\end{isadef}

\section{Stitching Port Graph Interfaces}
\label{app:seqInterfaceEdges}

In Section~\ref{sec:port_graphs/mech/seq} we make use of \isa{seqInterfaceEdges} to define port graph sequencing.
Our description of that function maps onto the Isabelle definitions as follows:
\begin{isadef}[Gathering edges going into output ports]{isa:edgesByOverallTo}
  \isacomm{primrec}\ edgesByOpenTo\ {\isacharcolon}{\isacharcolon}\ {\isachardoublequoteopen}{\isacharparenleft}\isatv{s}{\isacharcomma}\ \isatv{a}{\isacharcomma}\ \isatv{p}{\isacharparenright}\ edge\ list\isanewline
\isaindent{\isacomm{primrec}\ edgesByOpenTo\ }{\isasymRightarrow}\ {\isacharparenleft}{\isacharparenleft}\isatv{s}{\isacharcomma}\ \isatv{a}{\isacharparenright}\ port{\isacharcomma}\ {\isacharparenleft}\isatv{s}{\isacharcomma}\ \isatv{a}{\isacharcomma}\ \isatv{p}{\isacharparenright}\ edge\ list{\isacharparenright}\ mapping{\isachardoublequoteclose}\isanewline
\ \ \isaOcomm{where}\isanewline
\ \ \ \ {\isachardoublequoteopen}\isafv{edgesByOpenTo}\ {\isacharbrackleft}{\isacharbrackright}\ {\isacharequal}\ Mapping{\isachardot}empty{\isachardoublequoteclose}\isanewline
\ \ {\isacharbar}\ {\isachardoublequoteopen}\isafv{edgesByOpenTo}\ {\isacharparenleft}\isabv{e}{\isacharhash}\isabv{es}{\isacharparenright}\ {\isacharequal}\isanewline
\ \ \ \ {\isacharparenleft}\ \isanotation{if}\ place{\isacharunderscore}open\ {\isacharparenleft}edge{\isacharunderscore}to\ \isabv{e}{\isacharparenright}\isanewline
\ \ \ \ \ \ \ \ \isanotation{then}\ Mapping{\isachardot}map{\isacharunderscore}default\isanewline
\isaindent{\ \ \ \ \ \ \ \ \isanotation{then}\ Ma}{\isacharparenleft}place{\isacharunderscore}port\ {\isacharparenleft}edge{\isacharunderscore}to\ \isabv{e}{\isacharparenright}{\isacharparenright}\ Nil\ {\isacharparenleft}Cons\ \isabv{e}{\isacharparenright}\ {\isacharparenleft}\isafv{edgesByOpenTo}\ \isabv{es}{\isacharparenright}\isanewline
\ \ \ \ \ \ \ \ \isanotation{else}\ \isafv{edgesByOpenTo}\ \isabv{es}{\isacharparenright}{\isachardoublequoteclose}

\end{isadef}
\begin{isadef}[Gathering edges going from input ports]{isa:edgesByOverallFrom}
  \isacomm{primrec}\ edgesByOpenFrom\ {\isacharcolon}{\isacharcolon}\ {\isachardoublequoteopen}{\isacharparenleft}\isatv{s}{\isacharcomma}\ \isatv{a}{\isacharcomma}\ \isatv{p}{\isacharparenright}\ edge\ list\isanewline
\isaindent{\isacomm{primrec}\ edgesByOpenFrom\ }{\isasymRightarrow}\ {\isacharparenleft}{\isacharparenleft}\isatv{s}{\isacharcomma}\ \isatv{a}{\isacharparenright}\ port{\isacharcomma}\ {\isacharparenleft}\isatv{s}{\isacharcomma}\ \isatv{a}{\isacharcomma}\ \isatv{p}{\isacharparenright}\ edge\ list{\isacharparenright}\ mapping{\isachardoublequoteclose}\isanewline
\ \ \isaOcomm{where}\isanewline
\ \ \ \ {\isachardoublequoteopen}\isafv{edgesByOpenFrom}\ {\isacharbrackleft}{\isacharbrackright}\ {\isacharequal}\ Mapping{\isachardot}empty{\isachardoublequoteclose}\isanewline
\ \ {\isacharbar}\ {\isachardoublequoteopen}\isafv{edgesByOpenFrom}\ {\isacharparenleft}\isabv{e}{\isacharhash}\isabv{es}{\isacharparenright}\ {\isacharequal}\isanewline
\ \ \ \ {\isacharparenleft}\ \isanotation{if}\ place{\isacharunderscore}open\ {\isacharparenleft}edge{\isacharunderscore}from\ \isabv{e}{\isacharparenright}\isanewline
\ \ \ \ \ \ \ \ \isanotation{then}\ Mapping{\isachardot}map{\isacharunderscore}default\isanewline
\isaindent{\ \ \ \ \ \ \ \ \isanotation{then}\ Ma}{\isacharparenleft}place{\isacharunderscore}port\ {\isacharparenleft}edge{\isacharunderscore}from\ \isabv{e}{\isacharparenright}{\isacharparenright}\ Nil\ {\isacharparenleft}Cons\ \isabv{e}{\isacharparenright}\ {\isacharparenleft}\isafv{edgesByOpenFrom}\ \isabv{es}{\isacharparenright}\isanewline
\ \ \ \ \ \ \ \ \isanotation{else}\ \isafv{edgesByOpenFrom}\ \isabv{es}{\isachardoublequoteclose}

\end{isadef}
\begin{isadef}[Connect all combinations from two lists of places]{isa:allEdges}
  \isacomm{primrec}\isamarkupfalse%
\ allEdges\ {\isacharcolon}{\isacharcolon}\ {\isachardoublequoteopen}{\isacharparenleft}\isatv{s}{\isacharcomma}\ \isatv{a}{\isacharcomma}\ \isatv{p}{\isacharparenright}\ place\ list\ {\isasymRightarrow}\ {\isacharparenleft}\isatv{s}{\isacharcomma}\ \isatv{a}{\isacharcomma}\ \isatv{p}{\isacharparenright}\ place\ list\ {\isasymRightarrow}\ {\isacharparenleft}\isatv{s}{\isacharcomma}\ \isatv{a}{\isacharcomma}\ \isatv{p}{\isacharparenright}\ edge\ list{\isachardoublequoteclose}\isanewline
\ \ \isaOcomm{where}\isanewline
\ \ \ \ {\isachardoublequoteopen}\isafv{allEdges}\ {\isacharbrackleft}{\isacharbrackright}\ \isabv{ts}\ {\isacharequal}\ {\isacharbrackleft}{\isacharbrackright}{\isachardoublequoteclose}\isanewline
\ \ {\isacharbar}\ {\isachardoublequoteopen}\isafv{allEdges}\ {\isacharparenleft}\isabv{f}{\isacharhash}\isabv{fs}{\isacharparenright}\ \isabv{ts}\ {\isacharequal}\ map\ {\isacharparenleft}{\isasymlambda}\isabv{t}{\isachardot}\ Edge\ \isabv{f}\ \isabv{t}{\isacharparenright}\ \isabv{ts}\ {\isacharat}\ \isafv{allEdges}\ \isabv{fs}\ \isabv{ts}{\isachardoublequoteclose}

\end{isadef}
\begin{isadef}[Pair up place lists by shared port and connect them]{isa:edgesFromPortMapping}
  \isacomm{primrec}\ edgesFromPortMapping\ {\isacharcolon}{\isacharcolon}\ {\isachardoublequoteopen}\ {\isacharparenleft}\isatv{s}\ {\isacharcolon}{\isacharcolon}\ side{\isacharunderscore}in{\isacharunderscore}out{\isacharcomma}\ \isatv{a}{\isacharparenright}\ port\ list\isanewline
\isaindent{\isacomm{primrec}\ edgesFromPortMapping\ }{\isasymRightarrow}\ {\isacharparenleft}{\isacharparenleft}\isatv{s}{\isacharcomma}\ \isatv{a}{\isacharparenright}\ port{\isacharcomma}\ {\isacharparenleft}\isatv{s}{\isacharcomma}\ \isatv{a}{\isacharcomma}\ \isatv{p}{\isacharparenright}\ place\ list{\isacharparenright}\ mapping\isanewline
\isaindent{\isacomm{primrec}\ edgesFromPortMapping\ }{\isasymRightarrow}\ {\isacharparenleft}{\isacharparenleft}\isatv{s}{\isacharcomma}\ \isatv{a}{\isacharparenright}\ port{\isacharcomma}\ {\isacharparenleft}\isatv{s}{\isacharcomma}\ \isatv{a}{\isacharcomma}\ \isatv{p}{\isacharparenright}\ place\ list{\isacharparenright}\ mapping\isanewline
\isaindent{\isacomm{primrec}\ edgesFromPortMapping\ }{\isasymRightarrow}\ {\isacharparenleft}\isatv{s}{\isacharcomma}\ \isatv{a}{\isacharcomma}\ \isatv{p}{\isacharparenright}\ edge\ list{\isachardoublequoteclose}\isanewline
\ \ \isaOcomm{where}\isanewline
\ \ \ \ {\isachardoublequoteopen}\isafv{edgesFromPortMapping}\ {\isacharbrackleft}{\isacharbrackright}\ \isabv{x\ y}\ {\isacharequal}\ {\isacharbrackleft}{\isacharbrackright}{\isachardoublequoteclose}\isanewline
\ \ {\isacharbar}\ {\isachardoublequoteopen}\isafv{edgesFromPortMapping}\ {\isacharparenleft}\isabv{p}{\isacharhash}\isabv{ps}{\isacharparenright}\ \isabv{x\ y}\ {\isacharequal}\isanewline
\ \ \ \ {\isacharparenleft}\ \isanotation{case}\ Mapping{\isachardot}lookup\ \isabv{x}\ {\isacharparenleft}portSetSide\ Out\ \isabv{p}{\isacharparenright}\ \isanotation{of}\isanewline
\ \ \ \ \ \ \ \ None\ {\isasymRightarrow}\ {\isacharbrackleft}{\isacharbrackright}\isanewline
\ \ \ \ \ \ {\isacharbar}\ Some\ \isabv{xs}\ {\isasymRightarrow}\ {\isacharparenleft}\isanotation{case}\ Mapping{\isachardot}lookup\ \isabv{y}\ {\isacharparenleft}portSetSide\ In\ \isabv{p}{\isacharparenright}\ \isanotation{of}\isanewline
\ \ \ \ \ \ \ \ \ \ None\ {\isasymRightarrow}\ {\isacharbrackleft}{\isacharbrackright}\isanewline
\ \ \ \ \ \ \ \ {\isacharbar}\ Some\ \isabv{ys}\ {\isasymRightarrow}\ allEdges\ \isabv{xs\ ys}{\isacharparenright}{\isacharparenright}\isanewline
\ \ \ \ {\isacharat}\ \isafv{edgesFromPortMapping}\ \isabv{ps\ x\ y}{\isachardoublequoteclose}

\end{isadef}
\begin{isadef}[Connect edges by shared interface ports]{isa:seqInterfaceEdges}
  \isacomm{fun}\ seqInterfaceEdges\ {\isacharcolon}{\isacharcolon}\ {\isachardoublequoteopen}{\isacharparenleft}\isatv{s}\ {\isacharcolon}{\isacharcolon}\ side{\isacharunderscore}in{\isacharunderscore}out{\isacharcomma}\ \isatv{a}{\isacharcomma}\ \isatv{p}{\isacharcomma}\ \isatv{l}{\isacharparenright}\ port{\isacharunderscore}graph\isanewline
\isaindent{\isacommand{fun}\ seqInterfaceEdges\ }{\isasymRightarrow}\ {\isacharparenleft}\isatv{s}{\isacharcomma}\ \isatv{a}{\isacharcomma}\ \isatv{p}{\isacharcomma}\ \isatv{l}{\isacharparenright}\ port{\isacharunderscore}graph\isanewline
\isaindent{\isacommand{fun}\ seqInterfaceEdges\ }{\isasymRightarrow}\ {\isacharparenleft}\isatv{s}{\isacharcomma}\ \isatv{a}{\isacharcomma}\ \isatv{p}{\isacharparenright}\ edge\ list{\isachardoublequoteclose}\isanewline
\ \ \isaOcomm{where}\ {\isachardoublequoteopen}\isafv{seqInterfaceEdges}\ \isabv{x\ y}\ {\isacharequal}\isanewline
\ \ remdups\isanewline
\ \ \ \ {\isacharparenleft}\ edgesFromPortMapping\ {\isacharparenleft}filter\ {\isacharparenleft}{\isasymlambda}\isabv{p}{\isachardot}\ port{\isachardot}side\ \isabv{p}\ {\isacharequal}\ Out{\isacharparenright}\ {\isacharparenleft}pg{\isacharunderscore}ports\ \isabv{x}{\isacharparenright}{\isacharparenright}\isanewline
\ \ \ \ \ \ {\isacharparenleft}\ Mapping{\isachardot}map{\isacharunderscore}values\ {\isacharparenleft}{\isasymlambda}\isabv{k}{\isachardot}\ map\ edge{\isacharunderscore}from{\isacharparenright}\ {\isacharparenleft}edgesByOpenTo\ {\isacharparenleft}pg{\isacharunderscore}edges\ \isabv{x}{\isacharparenright}{\isacharparenright}{\isacharparenright}\isanewline
\ \ \ \ \ \ {\isacharparenleft}\ Mapping{\isachardot}map{\isacharunderscore}values\ {\isacharparenleft}{\isasymlambda}\isabv{k}{\isachardot}\ map\ edge{\isacharunderscore}to{\isacharparenright}\ {\isacharparenleft}edgesByOpenFrom\ {\isacharparenleft}pg{\isacharunderscore}edges\ \isabv{y}{\isacharparenright}{\isacharparenright}{\isacharparenright}{\isacharparenright}{\isachardoublequoteclose}%

\end{isadef}

\section{Conversion of Port Graphs to ELK JSON}
\label{app:elk_export}

In Section~\ref{sec:port_graphs/mech/export} we discuss our ability to visualise port graphs with the Eclipse Layout Kernel by converting them into JSON\footnote{\url{https://www.eclipse.org/elk/documentation/tooldevelopers/graphdatastructure/jsonformat.html}}.
The full definitions of the conversions involved are as follows:
\begin{isadef}[Datatype representing port sides that ELK supports]{isa:elk_side}
  \isacomm{datatype}\ elk{\isacharunderscore}side\ {\isacharequal}\ UNDEFINED\ {\isacharbar}\ NORTH\ {\isacharbar}\ EAST\ {\isacharbar}\ SOUTH\ {\isacharbar}\ WEST

\end{isadef}
\begin{isadef}[Converting ELK sides to their representation for JSON]{isa:elkSideToString}
  \isacomm{primrec}\ elkSideToString\ {\isacharcolon}{\isacharcolon}\ {\isachardoublequoteopen}elk{\isacharunderscore}side\ {\isasymRightarrow}\ String{\isachardot}literal{\isachardoublequoteclose}\ \isaOcomm{where}\isanewline
\ \ \ \ {\isachardoublequoteopen}\isafv{elkSideToString}\ UNDEFINED\ {\isacharequal}\ \isaString{UNDEFINED}{\isachardoublequoteclose}\isanewline
\ \ {\isacharbar}\ {\isachardoublequoteopen}\isafv{elkSideToString}\ NORTH\ {\isacharequal}\ \isaString{NORTH}{\isachardoublequoteclose}\isanewline
\ \ {\isacharbar}\ {\isachardoublequoteopen}\isafv{elkSideToString}\ EAST\ {\isacharequal}\ \isaString{EAST}{\isachardoublequoteclose}\isanewline
\ \ {\isacharbar}\ {\isachardoublequoteopen}\isafv{elkSideToString}\ SOUTH\ {\isacharequal}\ \isaString{SOUTH}{\isachardoublequoteclose}\isanewline
\ \ {\isacharbar}\ {\isachardoublequoteopen}\isafv{elkSideToString}\ WEST\ {\isacharequal}\ \isaString{WEST}{\isachardoublequoteclose}%

\end{isadef}
\begin{isadef}[Default assignment of input and output to ELK sides]{isa:sideInOutToELK}
  \isacomm{fun}\ sideInOutToELK\ {\isacharcolon}{\isacharcolon}\ {\isachardoublequoteopen}\isapars{\isatv{s}\ {\isacharcolon}{\isacharcolon}\ side{\isacharunderscore}in{\isacharunderscore}out}\ {\isasymRightarrow}\ elk{\isacharunderscore}side{\isachardoublequoteclose}\isanewline
\ \ \isaOcomm{where}\ {\isachardoublequoteopen}\isafv{sideInOutToELK}\ \isabv{x}\ {\isacharequal}\isanewline
\ \ \ \ {\isacharparenleft}\isanotation{if}\ \isabv{x}\ {\isacharequal}\ In\ \isanotation{then}\ WEST\ \isanotation{else\ if}\ \isabv{x}\ {\isacharequal}\ Out\ \isanotation{then}\ EAST\ \isanotation{else}\ UNDEFINED{\isacharparenright}{\isachardoublequoteclose}

\end{isadef}
\begin{isadef}[Locale for converting port graphs into ELK JSON]{isa:portGraphELK}
  \isacomm{locale}\isamarkupfalse%
\ portGraphELK\ {\isacharequal}\isanewline
\ \ \ \isaOcomm{fixes}\ \isafv{portSideToELK}\ {\isacharcolon}{\isacharcolon}\ {\isachardoublequoteopen}\isatv{s}\ {\isasymRightarrow}\ elk{\isacharunderscore}side{\isachardoublequoteclose}\isanewline
\ \ \ \ \isaOcomm{and}\ \isafv{pathToString}\ {\isacharcolon}{\isacharcolon}\ {\isachardoublequoteopen}\isatv{p}\ list\ {\isasymRightarrow}\ String{\isachardot}literal{\isachardoublequoteclose}\isanewline
\ \ \ \ \isaOcomm{and}\ \isafv{labelToString}\ {\isacharcolon}{\isacharcolon}\ {\isachardoublequoteopen}\isatv{l}\ {\isasymRightarrow}\ String{\isachardot}literal{\isachardoublequoteclose}

\end{isadef}
\begin{isadef}[Converting ground ports into ELK JSON]{isa:groundPortToJSON}
  \isacomm{definition}\ groundPortToJSON\ {\isacharcolon}{\isacharcolon}\ {\isachardoublequoteopen}\isatv{p}\ list\ {\isasymRightarrow}\ nat\ {\isasymRightarrow}\ {\isacharparenleft}\isatv{s}{\isacharcomma}\ \isatv{a}{\isacharparenright}\ port\isanewline
\isaindent{\isacomm{definition}\ groundPortToJSON\ }{\isasymRightarrow}\ {\isacharparenleft}String{\isachardot}literal{\isacharcomma}\ int{\isacharparenright}\ json{\isachardoublequoteclose}\isanewline
\ \ \isaOcomm{where}\ {\isachardoublequoteopen}\isafv{groundPortToJSON}\ \isabv{prefix\ maxIdx\ p}\ {\isacharequal}\isanewline
\ \ OBJECT\ {\isacharbrackleft}\isanewline
\ \ \ \ {\isacharparenleft}\isaString{id}{\isacharcomma}\ STRING\ {\isacharparenleft}\isafv{pathToString}\ \isabv{prefix}\ {\isacharplus}\isanewline
\isaindent{\ \ \ \ {\isacharparenleft}\isaString{id}{\isacharcomma}\ STRING\ {\isacharparenleft}}elkSideToString\ {\isacharparenleft}\isafv{portSideToELK}\ {\isacharparenleft}port{\isachardot}side\ \isabv{p}{\isacharparenright}{\isacharparenright}\ {\isacharplus}\isanewline
\isaindent{\ \ \ \ {\isacharparenleft}\isaString{id}{\isacharcomma}\ STRING\ {\isacharparenleft}}String{\isachardot}implode\ {\isacharparenleft}show\ {\isacharparenleft}port{\isachardot}index\ \isabv{p}{\isacharparenright}{\isacharparenright}{\isacharparenright}{\isacharparenright}\isanewline
\ \ {\isacharcomma}\ {\isacharparenleft}\isaString{properties}{\isacharcomma}\ OBJECT\ {\isacharbrackleft}\isanewline
\ \ \ \ \ \ {\isacharparenleft}\isaString{port{\isachardot}side}{\isacharcomma}\ STRING\ {\isacharparenleft}elkSideToString\ {\isacharparenleft}\isafv{portSideToELK}\ {\isacharparenleft}port{\isachardot}side\ \isabv{p}{\isacharparenright}{\isacharparenright}{\isacharparenright}{\isacharparenright}\isanewline
\ \ \ \ {\isacharcomma}\ {\isacharparenleft}\isaString{port{\isachardot}index}{\isacharcomma}\ NUMBER\ {\isacharparenleft}int\ {\isacharparenleft}\isanewline
\isaindent{\ \ \ \ {\isacharcomma}\ {\isacharparenleft}S}\isanotation{if}\ \isafv{portSideToELK}\ {\isacharparenleft}port{\isachardot}side\ \isabv{p}{\isacharparenright}\ {\isasymin}\ \isabraces{WEST{\isacharcomma}\ SOUTH}\isanewline
\isaindent{\ \ \ \ {\isacharcomma}\ {\isacharparenleft}Sif}\isanotation{then}\ \isabv{maxIdx}\ {\isacharminus}\ port{\isachardot}index\ \isabv{p}\isanewline
\isaindent{\ \ \ \ {\isacharcomma}\ {\isacharparenleft}Sif}\isanotation{else}\ port{\isachardot}index\ \isabv{p}{\isacharparenright}{\isacharparenright}{\isacharparenright}\isanewline
\ \ \ \ {\isacharbrackright}{\isacharparenright}\isanewline
\ \ {\isacharcomma}\ {\isacharparenleft}\isaString{width}{\isacharcomma}\ NUMBER\ {\isadigit{1}}{\isadigit{0}}{\isacharparenright}\isanewline
\ \ {\isacharcomma}\ {\isacharparenleft}\isaString{height}{\isacharcomma}\ NUMBER\ {\isadigit{1}}{\isadigit{0}}{\isacharparenright}{\isacharbrackright}{\isachardoublequoteclose}

\end{isadef}
\noindent(Note that we may need to invert the index, because ELK numbers ports clockwise while we number them from the top down and from left to right.)
\begin{isadef}[Converting nodes into ELK JSON]{isa:nodeToJSON}
  \isacomm{definition}\ nodeToJSON\ {\isacharcolon}{\isacharcolon}\ {\isachardoublequoteopen}\isatv{p}\ list\ {\isasymRightarrow}\ {\isacharparenleft}\isatv{s}{\isacharcomma}\ \isatv{a}{\isacharcomma}\ \isatv{p}{\isacharcomma}\ \isatv{l}{\isacharparenright}\ node\ {\isasymRightarrow}\ {\isacharparenleft}String{\isachardot}literal{\isacharcomma}\ int{\isacharparenright}\ json{\isachardoublequoteclose}\isanewline
\ \ \isaOcomm{where}\ {\isachardoublequoteopen}\isafv{nodeToJSON}\ \isabv{prefix\ n}\ {\isacharequal}\isanewline
\ \ OBJECT\ {\isacharbrackleft}\isanewline
\ \ \ \ {\isacharparenleft}\isaString{id}{\isacharcomma}\ STRING\ {\isacharparenleft}\isafv{pathToString}\ {\isacharparenleft}\isabv{prefix}\ {\isacharat}\ node{\isacharunderscore}name\ \isabv{n}{\isacharparenright}{\isacharparenright}{\isacharparenright}\isanewline
\ \ {\isacharcomma}\ {\isacharparenleft}\isaString{width}{\isacharcomma}\ NUMBER\ {\isadigit{6}}{\isadigit{0}}{\isacharparenright}\isanewline
\ \ {\isacharcomma}\ {\isacharparenleft}\isaString{height}{\isacharcomma}\ NUMBER\ {\isadigit{6}}{\isadigit{0}}{\isacharparenright}\isanewline
\ \ {\isacharcomma}\ {\isacharparenleft}\isaString{properties}{\isacharcomma}\ OBJECT\ {\isacharbrackleft}\isanewline
\ \ \ \ \ \ {\isacharparenleft}\isaString{portConstraints}{\isacharcomma}\ STRING\ \isaString{FIXED{\isacharunderscore}ORDER}{\isacharparenright}\isanewline
\ \ \ \ {\isacharcomma}\ {\isacharparenleft}\isaString{nodeLabels{\isachardot}placement}{\isacharcomma}\isanewline
\isaindent{\ \ \ \ {\isacharcomma}\ {\isacharparenleft}S}STRING\ \isaString{{\isacharbrackleft}H\_CENTER{\isacharcomma}\ V\_CENTER{\isacharcomma}\ INSIDE{\isacharbrackright}}{\isacharparenright}{\isacharbrackright}{\isacharparenright}\isanewline
\ \ {\isacharcomma}\ {\isacharparenleft}\isaString{labels}{\isacharcomma}\ ARRAY\ {\isacharbrackleft}OBJECT\ {\isacharbrackleft}\isanewline
\ \ \ \ \ \ {\isacharparenleft}\isaString{text}{\isacharcomma}\ STRING\ {\isacharparenleft}\isafv{labelToString}\ {\isacharparenleft}node{\isacharunderscore}label\ \isabv{n}{\isacharparenright}{\isacharparenright}{\isacharparenright}\isanewline
\ \ \ \ {\isacharcomma}\ {\isacharparenleft}\isaString{id}{\isacharcomma}\ STRING\ {\isacharparenleft}\isafv{pathToString}\ {\isacharparenleft}\isabv{prefix}\ {\isacharat}\ node{\isacharunderscore}name\ \isabv{n}{\isacharparenright}\ {\isacharplus}\ \isaString{\_label}{\isacharparenright}{\isacharparenright}\isanewline
\ \ \ \ {\isacharcomma}\ {\isacharparenleft}\isaString{width}{\isacharcomma}\ NUMBER\ {\isadigit{2}}{\isadigit{0}}{\isacharparenright}\isanewline
\ \ \ \ {\isacharcomma}\ {\isacharparenleft}\isaString{height}{\isacharcomma}\ NUMBER\ {\isadigit{2}}{\isadigit{0}}{\isacharparenright}{\isacharbrackright}{\isacharbrackright}{\isacharparenright}\isanewline
\ \ {\isacharcomma}\ {\isacharparenleft}\isaString{ports}{\isacharcomma}\ ARRAY\isanewline
\isaindent{\ \ {\isacharcomma}\ {\isacharparenleft}S}{\isacharparenleft}map\ {\isacharparenleft}{\isasymlambda}\isabv{p}{\isachardot}\ groundPortToJSON\isanewline
\isaindent{\ \ {\isacharcomma}\ {\isacharparenleft}S{\isacharparenleft}map\ {\isacharparenleft}{\isasymlambda}\isabv{p}{\isachardot}\ g}{\isacharparenleft}\isabv{prefix}\ {\isacharat}\ node{\isacharunderscore}name\ \isabv{n}{\isacharparenright}\isanewline
\isaindent{\ \ {\isacharcomma}\ {\isacharparenleft}S{\isacharparenleft}map\ {\isacharparenleft}{\isasymlambda}\isabv{p}{\isachardot}\ g}{\isacharparenleft}length\ {\isacharparenleft}filter\ {\isacharparenleft}{\isasymlambda}\isabv{x}{\isachardot}\ port{\isachardot}side\ \isabv{x}\ {\isacharequal}\ port{\isachardot}side\ \isabv{p}{\isacharparenright}\ {\isacharparenleft}node{\isacharunderscore}ports\ \isabv{n}{\isacharparenright}{\isacharparenright}\ {\isacharminus}\ {\isadigit{1}}{\isacharparenright}\isanewline
\isaindent{\ \ {\isacharcomma}\ {\isacharparenleft}S{\isacharparenleft}map\ {\isacharparenleft}{\isasymlambda}\isabv{p}{\isachardot}\ g}\isabv{p}{\isacharparenright}\isanewline
\isaindent{\ \ {\isacharcomma}\ {\isacharparenleft}S{\isacharparenleft}map\ }{\isacharparenleft}node{\isacharunderscore}ports\ \isabv{n}{\isacharparenright}{\isacharparenright}{\isacharparenright}{\isacharbrackright}{\isachardoublequoteclose}

\end{isadef}
\begin{isadef}[Converting an arbitrary place to its ID]{isa:placeToID}
    \isacomm{definition}\isamarkupfalse%
\ groundPlaceToID\ {\isacharcolon}{\isacharcolon}\ {\isachardoublequoteopen}\isatv{p}\ list\ {\isasymRightarrow}\ {\isacharparenleft}\isatv{s}{\isacharcomma}\ \isatv{a}{\isacharcomma}\ \isatv{p}{\isacharparenright}\ place\ {\isasymRightarrow}\ String{\isachardot}literal{\isachardoublequoteclose}\isanewline
\ \ \isaOcomm{where}\ {\isachardoublequoteopen}\isafv{groundPlaceToID}\ \isabv{prefix\ p}\ {\isacharequal}\isanewline
\ \ \ \ \isafv{pathToString}\ {\isacharparenleft}\isabv{prefix}\ {\isacharat}\ place{\isacharunderscore}name\ \isabv{p}{\isacharparenright}\ {\isacharplus}\isanewline
\ \ \ \ elkSideToString\ {\isacharparenleft}\isafv{portSideToELK}\ {\isacharparenleft}port{\isachardot}side\ {\isacharparenleft}place{\isacharunderscore}port\ \isabv{p}{\isacharparenright}{\isacharparenright}{\isacharparenright}\ {\isacharplus}\isanewline
\ \ \ \ String{\isachardot}implode\ {\isacharparenleft}show\ {\isacharparenleft}port{\isachardot}index\ {\isacharparenleft}place{\isacharunderscore}port\ \isabv{p}{\isacharparenright}{\isacharparenright}{\isacharparenright}{\isachardoublequoteclose}

  \item
    \isacomm{definition}\isamarkupfalse%
\ openPlaceToID\ {\isacharcolon}{\isacharcolon}\ {\isachardoublequoteopen}{\isacharparenleft}\isatv{s}{\isacharcomma}\ \isatv{a}{\isacharparenright}\ port\ {\isasymRightarrow}\ String{\isachardot}literal{\isachardoublequoteclose}\isanewline
\ \ \isaOcomm{where}\ {\isachardoublequoteopen}\isafv{openPlaceToID}\ \isabv{port}\ {\isacharequal}\isanewline
\ \ \isaString{Open}\ {\isacharplus}\isanewline
\ \ elkSideToString\ {\isacharparenleft}\isafv{portSideToELK}\ {\isacharparenleft}port{\isachardot}side\ \isabv{port}{\isacharparenright}{\isacharparenright}\ {\isacharplus}\isanewline
\ \ String{\isachardot}implode\ {\isacharparenleft}show\ {\isacharparenleft}port{\isachardot}index\ \isabv{port}{\isacharparenright}{\isacharparenright}{\isachardoublequoteclose}

  \item
    \isacomm{definition}\isamarkupfalse%
\ placeToID\ {\isacharcolon}{\isacharcolon}\ {\isachardoublequoteopen}\isatv{p}\ list\ {\isasymRightarrow}\ {\isacharparenleft}\isatv{s}{\isacharcomma}\ \isatv{a}{\isacharcomma}\ \isatv{p}{\isacharparenright}\ place\ {\isasymRightarrow}\ String{\isachardot}literal{\isachardoublequoteclose}\isanewline
\ \ \isaOcomm{where}\ {\isachardoublequoteopen}\isafv{placeToID}\ \isabv{prefix\ p}\ {\isacharequal}\ {\isacharparenleft}\ \isanotation{case}\ \isabv{p}\ \isanotation{of}\isanewline
\ \ \ \ \ \ GroundPort\ {\isacharunderscore}\ {\isasymRightarrow}\ groundPlaceToID\ \isabv{prefix\ p}\isanewline
\ \ \ \ {\isacharbar}\ OpenPort\ \isabv{port}\ {\isasymRightarrow}\ openPlaceToID\ \isabv{port}{\isacharparenright}{\isachardoublequoteclose}

\end{isadef}
\begin{isadef}[Converting edges into ELK JSON]{isa:edgeToJSON}
  \isacomm{definition}\isamarkupfalse%
\ edgeToJSON\ {\isacharcolon}{\isacharcolon}\ {\isachardoublequoteopen}\isatv{p}\ list\ {\isasymRightarrow}\ {\isacharparenleft}\isatv{s}{\isacharcomma}\ \isatv{a}{\isacharcomma}\ \isatv{p}{\isacharparenright}\ edge\ {\isasymRightarrow}\ {\isacharparenleft}String{\isachardot}literal{\isacharcomma}\ int{\isacharparenright}\ json{\isachardoublequoteclose}\isanewline
\ \ \isaOcomm{where}\ {\isachardoublequoteopen}\isafv{edgeToJSON}\ \isabv{prefix\ e}\ {\isacharequal}\isanewline
\ \ OBJECT\ {\isacharbrackleft}\isanewline
\ \ \ \ {\isacharparenleft}\isaString{id}{\isacharcomma}\ STRING\ {\isacharparenleft}placeToID\ prefix\ {\isacharparenleft}edge{\isacharunderscore}from\ \isabv{e}{\isacharparenright}\ {\isacharplus}\isanewline
\isaindent{\ \ \ \ {\isacharparenleft}\isaString{id}{\isacharcomma}\ STRING\ {\isacharparenleft}}\isaString{{\isacharminus}}\ {\isacharplus}\isanewline
\isaindent{\ \ \ \ {\isacharparenleft}\isaString{id}{\isacharcomma}\ STRING\ {\isacharparenleft}}placeToID\ prefix\ {\isacharparenleft}edge{\isacharunderscore}to\ \isabv{e}{\isacharparenright}{\isacharparenright}{\isacharparenright}\isanewline
\ \ {\isacharcomma}\ {\isacharparenleft}\isaString{sources}{\isacharcomma}\ ARRAY\ {\isacharbrackleft}STRING\ {\isacharparenleft}placeToID\ prefix\ {\isacharparenleft}edge{\isacharunderscore}from\ \isabv{e}{\isacharparenright}{\isacharparenright}{\isacharbrackright}{\isacharparenright}\isanewline
\ \ {\isacharcomma}\ {\isacharparenleft}\isaString{targets}{\isacharcomma}\ ARRAY\ {\isacharbrackleft}STRING\ {\isacharparenleft}placeToID\ prefix\ {\isacharparenleft}edge{\isacharunderscore}to\ \isabv{e}{\isacharparenright}{\isacharparenright}{\isacharbrackright}{\isacharparenright}{\isacharbrackright}{\isachardoublequoteclose}

\end{isadef}
\begin{isadef}[Converting open ports into ELK JSON]{isa:openPortToJSON}
  \isacomm{definition}\isamarkupfalse%
\ openPortToJSON\ {\isacharcolon}{\isacharcolon}\ {\isachardoublequoteopen}nat\ {\isasymRightarrow}\ {\isacharparenleft}\isatv{s}{\isacharcomma}\ \isatv{a}{\isacharparenright}\ port\ {\isasymRightarrow}\ {\isacharparenleft}String{\isachardot}literal{\isacharcomma}\ int{\isacharparenright}\ json{\isachardoublequoteclose}\isanewline
\ \ \isaOcomm{where}\ {\isachardoublequoteopen}\isafv{openPortToJSON}\ \isabv{maxIdx\ p}\ {\isacharequal}\isanewline
\ \ OBJECT\ {\isacharbrackleft}\isanewline
\ \ \ \ {\isacharparenleft}\isaString{id}{\isacharcomma}\ STRING\ {\isacharparenleft}openPlaceToID\ \isabv{p}{\isacharparenright}{\isacharparenright}\isanewline
\ \ {\isacharcomma}\ {\isacharparenleft}\isaString{properties}{\isacharcomma}\ OBJECT\ {\isacharbrackleft}\isanewline
\ \ \ \ \ \ {\isacharparenleft}\isaString{port{\isachardot}side}{\isacharcomma}\ STRING\ {\isacharparenleft}elkSideToString\ {\isacharparenleft}\isafv{portSideToELK}\ {\isacharparenleft}port{\isachardot}side\ \isabv{p}{\isacharparenright}{\isacharparenright}{\isacharparenright}{\isacharparenright}\isanewline
\ \ \ \ {\isacharcomma}\ {\isacharparenleft}\isaString{port{\isachardot}index}{\isacharcomma}\ NUMBER\ {\isacharparenleft}int\ {\isacharparenleft}\isanewline
\isaindent{\ \ \ \ {\isacharcomma}\ {\isacharparenleft}S}\isanotation{if}\ \isafv{portSideToELK}\ {\isacharparenleft}port{\isachardot}side\ \isabv{p}{\isacharparenright}\ {\isasymin}\ \isabraces{WEST{\isacharcomma}\ SOUTH}\isanewline
\isaindent{\ \ \ \ {\isacharcomma}\ {\isacharparenleft}Sif}\isanotation{then}\ \isabv{maxIdx}\ {\isacharminus}\ port{\isachardot}index\ \isabv{p}\isanewline
\isaindent{\ \ \ \ {\isacharcomma}\ {\isacharparenleft}Sif}\isanotation{else}\ port{\isachardot}index\ \isabv{p}{\isacharparenright}{\isacharparenright}{\isacharparenright}\isanewline
\ \ \ \ {\isacharbrackright}{\isacharparenright}\isanewline
\ \ {\isacharcomma}\ {\isacharparenleft}\isaString{width}{\isacharcomma}\ NUMBER\ {\isadigit{1}}{\isadigit{0}}{\isacharparenright}\isanewline
\ \ {\isacharcomma}\ {\isacharparenleft}\isaString{height}{\isacharcomma}\ NUMBER\ {\isadigit{1}}{\isadigit{0}}{\isacharparenright}{\isacharbrackright}{\isachardoublequoteclose}

\end{isadef}
\begin{isadef}[Converting port graphs into ELK JSON]{isa:portGraphToJSON}
  \isacomm{definition}\ portGraphToJSON\ {\isacharcolon}{\isacharcolon}\ {\isachardoublequoteopen}\isatv{p}\ list\ {\isasymRightarrow}\ {\isacharparenleft}\isatv{s}{\isacharcomma}\ \isatv{a}{\isacharcomma}\ \isatv{p}{\isacharcomma}\ \isatv{l}{\isacharparenright}\ port{\isacharunderscore}graph\isanewline
\isaindent{\isacomm{definition}\ portGraphToJSON\ }{\isasymRightarrow}\ {\isacharparenleft}String{\isachardot}literal{\isacharcomma}\ int{\isacharparenright}\ json{\isachardoublequoteclose}\isanewline
\ \ \isaOcomm{where}\ {\isachardoublequoteopen}\isafv{portGraphToJSON}\ \isabv{prefix\ G}\ {\isacharequal}\isanewline
\ \ OBJECT\ {\isacharbrackleft}\isanewline
\ \ \ \ {\isacharparenleft}\isaString{id}{\isacharcomma}\ STRING\ {\isacharparenleft}\isafv{pathToString}\ \isabv{prefix}\ {\isacharplus}\ \isaString{Root}{\isacharparenright}{\isacharparenright}\isanewline
\ \ {\isacharcomma}\ {\isacharparenleft}\isaString{layoutOptions}{\isacharcomma}\ OBJECT\ {\isacharbrackleft}{\isacharparenleft}\isaString{algorithm}{\isacharcomma}\ STRING\ \isaString{layered}{\isacharparenright}{\isacharbrackright}{\isacharparenright}\isanewline
\ \ {\isacharcomma}\ {\isacharparenleft}\isaString{children}{\isacharcomma}\ ARRAY\ {\isacharparenleft}\isanewline
\isaindent{\ \ {\isacharcomma}\ {\isacharparenleft}S}map\ {\isacharparenleft}nodeToJSON\ \isabv{prefix}{\isacharparenright}\ {\isacharparenleft}pg{\isacharunderscore}nodes\ \isabv{G}{\isacharparenright}\ {\isacharat}\isanewline
\isaindent{\ \ {\isacharcomma}\ {\isacharparenleft}S}map\ {\isacharparenleft}{\isasymlambda}\isabv{p}{\isachardot}\ openPortToJSON\isanewline
\isaindent{\ \ {\isacharcomma}\ {\isacharparenleft}Smap\ {\isacharparenleft}{\isasymlambda}p{\isachardot}\ o}{\isacharparenleft}length\ {\isacharparenleft}filter\ {\isacharparenleft}{\isasymlambda}\isabv{x}{\isachardot}\ port{\isachardot}side\ \isabv{x}\ {\isacharequal}\ port{\isachardot}side\ \isabv{p}{\isacharparenright}\ {\isacharparenleft}pg{\isacharunderscore}ports\ \isabv{G}{\isacharparenright}{\isacharparenright}\ {\isacharminus}\ {\isadigit{1}}{\isacharparenright}\isanewline
\isaindent{\ \ {\isacharcomma}\ {\isacharparenleft}Smap\ {\isacharparenleft}{\isasymlambda}p{\isachardot}\ o}\isabv{p}{\isacharparenright}\isanewline
\isaindent{\ \ {\isacharcomma}\ {\isacharparenleft}Smap\ }{\isacharparenleft}pg{\isacharunderscore}ports\ \isabv{G}{\isacharparenright}{\isacharparenright}{\isacharparenright}\isanewline
\ \ {\isacharcomma}\ {\isacharparenleft}\isaString{edges}{\isacharcomma}\ ARRAY\ {\isacharparenleft}map\ {\isacharparenleft}edgeToJSON\ \isabv{prefix}{\isacharparenright}\ {\isacharparenleft}pg{\isacharunderscore}edges\ \isabv{G}{\isacharparenright}{\isacharparenright}{\isacharparenright}{\isacharbrackright}{\isachardoublequoteclose}

\end{isadef}
\noindent(Note that we represent open ports with small nodes, because ELK does not support ports not attached to a node.)

\cbstart
\section{Processes with Port Graphs}
\label{app:pgDefined}

In Chapter~\ref{ch:port_graphs} we make frequent use of the predicate \isa{pgDefined} to exclude from consideration compositions that make use of non-deterministic or higher-order features, for which we do not define the port graph construction.
Here we give the predicate in full in Definition~\ref{isa:pgDefined}.
Note that we do not exclude the repeatable executable resource actions \isa{Repeat}, \isa{Close} and \isa{Once}, since those concern the copyable aspect of those resources.

\begin{isadef}[Predicate for processes that have port graphs]{isa:pgDefined}
  \isacomm{primrec}\isamarkupfalse%
\ pgDefined\ {\isacharcolon}{\isacharcolon}\ {\isachardoublequoteopen}{\isacharparenleft}\isatv{a}{\isacharcomma}\ \isatv{b}{\isacharcomma}\ \isatv{l}{\isacharcomma}\ \isatv{m}{\isacharparenright}\ process\ {\isasymRightarrow}\ bool{\isachardoublequoteclose}\ \isaOcomm{where}\isanewline
\ \ \ \ {\isachardoublequoteopen}\isafv{pgDefined}\ {\isacharparenleft}Primitive\ \isabv{ins\ outs\ l\ m}{\isacharparenright}\ {\isacharequal}\ True{\isachardoublequoteclose}\isanewline
\ \ {\isacharbar}\ {\isachardoublequoteopen}\isafv{pgDefined}\ {\isacharparenleft}Identity\ \isabv{a}{\isacharparenright}\ {\isacharequal}\ True{\isachardoublequoteclose}\isanewline
\ \ {\isacharbar}\ {\isachardoublequoteopen}\isafv{pgDefined}\ {\isacharparenleft}Swap\ \isabv{a\ b}{\isacharparenright}\ {\isacharequal}\ True{\isachardoublequoteclose}\isanewline
\ \ {\isacharbar}\ {\isachardoublequoteopen}\isafv{pgDefined}\ {\isacharparenleft}Seq\ \isabv{p\ q}{\isacharparenright}\ {\isacharequal}\ {\isacharparenleft}pgDefined\ \isabv{p}\ {\isasymand}\ pgDefined\ \isabv{q}{\isacharparenright}{\isachardoublequoteclose}\isanewline
\ \ {\isacharbar}\ {\isachardoublequoteopen}\isafv{pgDefined}\ {\isacharparenleft}Par\ \isabv{p\ q}{\isacharparenright}\ {\isacharequal}\ {\isacharparenleft}pgDefined\ \isabv{p}\ {\isasymand}\ pgDefined\ \isabv{q}{\isacharparenright}{\isachardoublequoteclose}\isanewline
\ \ {\isacharbar}\ {\isachardoublequoteopen}\isafv{pgDefined}\ {\isacharparenleft}Opt\ \isabv{p\ q}{\isacharparenright}\ {\isacharequal}\ False{\isachardoublequoteclose}\isanewline
\ \ {\isacharbar}\ {\isachardoublequoteopen}\isafv{pgDefined}\ {\isacharparenleft}InjectL\ \isabv{a\ b}{\isacharparenright}\ {\isacharequal}\ False{\isachardoublequoteclose}\isanewline
\ \ {\isacharbar}\ {\isachardoublequoteopen}\isafv{pgDefined}\ {\isacharparenleft}InjectR\ \isabv{a\ b}{\isacharparenright}\ {\isacharequal}\ False{\isachardoublequoteclose}\isanewline
\ \ {\isacharbar}\ {\isachardoublequoteopen}\isafv{pgDefined}\ {\isacharparenleft}OptDistrIn\ \isabv{a\ b\ c}{\isacharparenright}\ {\isacharequal}\ False{\isachardoublequoteclose}\isanewline
\ \ {\isacharbar}\ {\isachardoublequoteopen}\isafv{pgDefined}\ {\isacharparenleft}OptDistrOut\ \isabv{a\ b\ c}{\isacharparenright}\ {\isacharequal}\ False{\isachardoublequoteclose}\isanewline
\ \ {\isacharbar}\ {\isachardoublequoteopen}\isafv{pgDefined}\ {\isacharparenleft}Duplicate\ \isabv{a}{\isacharparenright}\ {\isacharequal}\ True{\isachardoublequoteclose}\isanewline
\ \ {\isacharbar}\ {\isachardoublequoteopen}\isafv{pgDefined}\ {\isacharparenleft}Erase\ \isabv{a}{\isacharparenright}\ {\isacharequal}\ True{\isachardoublequoteclose}\isanewline
\ \ {\isacharbar}\ {\isachardoublequoteopen}\isafv{pgDefined}\ {\isacharparenleft}Represent\ \isabv{p}{\isacharparenright}\ {\isacharequal}\ False{\isachardoublequoteclose}\isanewline
\ \ {\isacharbar}\ {\isachardoublequoteopen}\isafv{pgDefined}\ {\isacharparenleft}Apply\ \isabv{a\ b}{\isacharparenright}\ {\isacharequal}\ False{\isachardoublequoteclose}\isanewline
\ \ {\isacharbar}\ {\isachardoublequoteopen}\isafv{pgDefined}\ {\isacharparenleft}Repeat\ \isabv{a\ b}{\isacharparenright}\ {\isacharequal}\ True{\isachardoublequoteclose}\isanewline
\ \ {\isacharbar}\ {\isachardoublequoteopen}\isafv{pgDefined}\ {\isacharparenleft}Close\ \isabv{a\ b}{\isacharparenright}\ {\isacharequal}\ True{\isachardoublequoteclose}\isanewline
\ \ {\isacharbar}\ {\isachardoublequoteopen}\isafv{pgDefined}\ {\isacharparenleft}Once\ \isabv{a\ b}{\isacharparenright}\ {\isacharequal}\ True{\isachardoublequoteclose}\isanewline
\ \ {\isacharbar}\ {\isachardoublequoteopen}\isafv{pgDefined}\ {\isacharparenleft}Forget\ \isabv{a}{\isacharparenright}\ {\isacharequal}\ True{\isachardoublequoteclose}

\end{isadef}
\cbend

\section{List-Based Process Compositions}
\label{app:process-list}

To make definitions of larger process compositions more concise, we define functions for composing a list of processes in sequence or in parallel:
\begin{isadef}[Composing a list of processes in sequence]{isa:seq_process_list}
  \isacomm{primrec}\isamarkupfalse%
\ seq{\isacharunderscore}process{\isacharunderscore}list\ {\isacharcolon}{\isacharcolon}\ {\isachardoublequoteopen}{\isacharparenleft}\isatv{a}{\isacharcomma}\ \isatv{b}{\isacharcomma}\ \isatv{l}{\isacharcomma}\ \isatv{m}{\isacharparenright}\ process\ list\ {\isasymRightarrow}\ {\isacharparenleft}\isatv{a}{\isacharcomma}\ \isatv{b}{\isacharcomma}\ \isatv{l}{\isacharcomma}\ \isatv{m}{\isacharparenright}\ process{\isachardoublequoteclose}\isanewline
\ \ \isaOcomm{where}\isanewline
\ \ \ \ {\isachardoublequoteopen}\isafv{seq{\isacharunderscore}process{\isacharunderscore}list}\ {\isacharbrackleft}{\isacharbrackright}\ {\isacharequal}\ Identity\ Empty{\isachardoublequoteclose}\isanewline
\ \ {\isacharbar}\ {\isachardoublequoteopen}\isafv{seq{\isacharunderscore}process{\isacharunderscore}list}\ {\isacharparenleft}\isabv{x}\ {\isacharhash}\ \isabv{xs}{\isacharparenright}\ {\isacharequal}\ {\isacharparenleft}\isanotation{if}\ \isabv{xs}\ {\isacharequal}\ {\isacharbrackleft}{\isacharbrackright}\ \isanotation{then}\ \isabv{x}\ \isanotation{else}\ Seq\ \isabv{x}\ {\isacharparenleft}\isafv{seq{\isacharunderscore}process{\isacharunderscore}list}\ \isabv{xs}{\isacharparenright}{\isacharparenright}{\isachardoublequoteclose}

\end{isadef}

\begin{isadef}[Composing a list of processes in parallel]{isa:par_process_list}
  \isacomm{primrec}\isamarkupfalse%
\ par{\isacharunderscore}process{\isacharunderscore}list\ {\isacharcolon}{\isacharcolon}\ {\isachardoublequoteopen}{\isacharparenleft}\isatv{a}{\isacharcomma}\ \isatv{b}{\isacharcomma}\ \isatv{l}{\isacharcomma}\ \isatv{m}{\isacharparenright}\ process\ list\ {\isasymRightarrow}\ {\isacharparenleft}\isatv{a}{\isacharcomma}\ \isatv{b}{\isacharcomma}\ \isatv{l}{\isacharcomma}\ \isatv{m}{\isacharparenright}\ process{\isachardoublequoteclose}\isanewline
\ \ \isaOcomm{where}\isanewline
\ \ \ \ {\isachardoublequoteopen}\isafv{par{\isacharunderscore}process{\isacharunderscore}list}\ {\isacharbrackleft}{\isacharbrackright}\ {\isacharequal}\ Identity\ Empty{\isachardoublequoteclose}\isanewline
\ \ {\isacharbar}\ {\isachardoublequoteopen}\isafv{par{\isacharunderscore}process{\isacharunderscore}list}\ {\isacharparenleft}\isabv{x}\ {\isacharhash}\ \isabv{xs}{\isacharparenright}\ {\isacharequal}\ {\isacharparenleft}\isanotation{if}\ \isabv{xs}\ {\isacharequal}\ {\isacharbrackleft}{\isacharbrackright}\ \isanotation{then}\ \isabv{x}\ \isanotation{else}\ Par\ \isabv{x}\ {\isacharparenleft}\isafv{par{\isacharunderscore}process{\isacharunderscore}list}\ \isabv{xs}{\isacharparenright}{\isacharparenright}{\isachardoublequoteclose}

\end{isadef}

\cbstart
\section{Contingent Plan for Three Socks}
\label{app:three-socks-contingent}

In Section~\ref{sec:cases/three-socks} we describe a conformant plan for the problem and note the existence of a contingent variant.
The Haskell code for the composition representing this contingent plan is shown in Listing~\ref{lst:contingent}.

\begin{lstlisting}[label=lst:contingent,caption=Contingent plan for picking three matching socks,basicstyle=\footnotesize\ttfamily,columns=flexible,breaklines=true]
contingentPlan :: Process LRes () String ()
contingentPlan = seq_process_list
  [ Par (Identity (hidden # hidden)) pick
  , OptDistrIn (hidden # hidden) black white
  , Opt
      (seq_process_list
        [ Swap (hidden # hidden) black
        , Par (Identity (black # hidden)) pick
        , OptDistrIn (black # hidden) black white
        , Opt
            (seq_process_list
              [ Par (Identity black) (Swap hidden black)
              , Par (Identity (black # black)) (Forget hidden)
              , InjectL (black # black # anything) (white # white # anything)
              ])
            (seq_process_list
              [ Par (Identity black) (Swap hidden white)
              , Par (Identity (black # white)) pick
              , OptDistrIn (black # white) black white
              , Opt
                  (seq_process_list
                    [ Par (Identity black) (Swap white black)
                    , Par (Identity (black # black)) (Forget white)
                    , InjectL (black # black # anything) (white # white # anything)
                    ])
                  (seq_process_list
                    [ Swap black (white # white)
                    , Par (Identity (white # white)) (Forget black)
                    , InjectR (black # black # anything) (white # white # anything)
                    ])
              ])
        ])
      (seq_process_list
        [ Swap (hidden # hidden) white
        , Par (Identity (white # hidden)) pick
        , OptDistrIn (white # hidden) black white
        , Opt
            (seq_process_list
              [ Par (Identity white) (Swap hidden black)
              , Par (Identity (white # black)) pick
              , OptDistrIn (white # black) black white
              , Opt
                  (seq_process_list
                    [ Swap white (black # black)
                    , Par (Identity (black # black)) (Forget white)
                    , InjectL (black # black # anything) (white # white # anything)
                    ])
                  (seq_process_list
                    [ Par (Identity white) (Swap black white)
                    , Par (Identity (white # white)) (Forget black)
                    , InjectR (black # black # anything) (white # white # anything)
                    ])
              ])
            (seq_process_list
              [ Par (Identity white) (Swap hidden white)
              , Par (Identity (white # white)) (Forget hidden)
              , InjectR (black # black # anything) (white # white # anything)
              ])
        ])
  ]
\end{lstlisting}
\cbend

\ifstandalone
\bibliographystyle{plain}
\bibliography{references}
\fi

\end{document}
