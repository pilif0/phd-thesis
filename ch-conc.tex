\documentclass[class=smolathesis,crop=false]{standalone}

\begin{document}

\chapter{Conclusion}
\label{ch:conc}

In the preceding chapters we developed a framework for verified composition of processes based on linear resources.
We did so in the proof assistant Isabelle/HOL to ensure a high degree of confidence in the framework's properties and to allow for automated generation of executable code.

We connected our process compositions to two other concepts: linear logic deductions and port graphs.
The first focuses on the correspondence of deduction in linear logic to the structure of correct processes, while the second focuses on viewing large processes as collections of actions connected by their inputs and outputs.
We use both to demonstrate that valid process compositions manipulate resources correctly.

We also explored an extension of our framework to include probabilistic information in its description of non-determinism.
While the present outcome of this exploration is open to definite improvements, it did demonstrate how invaluable a proof assistant is in exploring changes to the foundations of a formal framework.
With its assistance, making such changes is significantly easier and, as a result, exploratory studies of formalised systems become more viable.

We demonstrated the application of our framework with a number of case studies in different domains.
While these do not cover the full breadth of possibility, they illustrate the framework's main features and suggest more complex uses.

\section{Future Work}
\label{sec:conc/future}

In this section we highlight some of the main threads of future work arising from this thesis.

\paragraph*{Process behaviour.}
A formalisation of behaviour for our process compositions would allow us to prove which of the latter, although syntactically distinct, represent the same behaviour.
This in turn would allow us to verify, for instance, behaviour-preserving simplifications of process compositions such as removing identity actions from sequential composition.

We see process port graphs from Section~\ref{sec:port_graphs/process} and the transition system for port graphs from Section~\ref{sec:port_graphs/trans} as providing a promising path towards formalising a notion of behaviour for our process compositions.
A future task for this thread lies in expanding our formalisation of port graphs so that they can capture non-determinism and higher-order processes.

\paragraph*{Connection to other formalisms.}
A complementary approach to process composition behaviour would be to formally connect them to other process formalisms, such as $\pi$-calculus or Petri nets.
We could then use their established theory and tools in our framework.
That includes making use of the their transition systems to characterise the behaviour of our process compositions.

In order to mechanise the connection, we would require mechanisations of those process formalisms in Isabelle/HOL.
There exists such a mechanisation for the $\pi$-calculus due to Bengtson~\cite{Pi_Calculus-AFP}.
As far as we are aware, there is no published mechanisation of Petri nets in Isabelle/HOL.

Both of these formalisms make use of connections between individual actions: as channels between agents in $\pi$-calculus and as places between transitions in Petri nets.
As such, we believe a significant part of formally connecting process compositions to them would be handled by process port graphs.
Then, beyond finding a suitable mechanisation of the target formalism, the major task would be connecting to it from port graphs.

\paragraph*{Probabilistic information.}
As noted, the outcome of our exploration of probabilistic resources in Chapter~\ref{ch:prob} is relatively preliminary.
While the extended framework can express the probabilistic information and make use of it in computing expected values from process compositions, it produces impractically complex resource expressions.
As such, one thread of future work would be about revising how probabilistic information can be better integrated into resources and process compositions in light of lessons learned through our exploration.

\paragraph*{Concrete tools.}
Mechanising our framework in Isabelle/HOL allows us to automatically generate executable code from it.
We make use of it to generate process diagrams (see Section~\ref{sec:proc/diag}) and in our model of manufacturing in Factorio to generate instructions for implementing a process as a factory (see Section~\ref{sec:cases/factorio/instr}).

There are more tools we could implement by utilising the verified code generated by Isabelle/HOL, boosting the practicality of our framework.
For instance, recall our definition of process port graphs (see Section~\ref{sec:port_graphs/process}) and our conversion of port graphs into a format that can be visualised by the Eclipse Layout Kernel and Sprotty (see Section~\ref{sec:port_graphs/mech/export}).
We could make use of the verified code for these to implement a visualisation tool tailored to process compositions.
Unlike our process diagrams, such a tool could be made interactive, allowing us to better inspect the process in question or even construct processes in a graphical environment.

We could additionally make use of the port graph transition system (see Section~\ref{sec:port_graphs/trans}) to identify enabled actions.
Then we could distinguish the relevant nodes in the visualisation, forming a kind of graphical checklist for the process.
It could be made interactive, allowing users to mark actions as completed to update the highlighted set of enabled actions.
We believe this would aid coordination in settings where multiple people cooperate to execute a large process.

Our ability to generate verified code in Scala, Haskell, OCaml and SML offers the opportunity to integrate with a wide variety of external frameworks while retaining the rigour of our formally verified approach.

\section{Concluding Remarks}
\label{sec:conc/remarks}

Our mechanised framework gives us a solid foundation for tools whose implementation of processes can be checked, so that the assistance they provide can be relied upon.
The language of process compositions is simple and avoids ambiguity, so that our expectations are clear both when modelling processes and when using those models.
Moreover, the verified code generated by Isabelle ensures that what the tools use is what we formalised.

Looking toward the future, we will use this foundation to build such tools and expand the range of domains where we have applied our framework.
Reaching more domains and modelling increasingly complex processes will allow us to truly push the framework to its limits, and then, with the assistance of Isabelle, expand those limits in a rigorous way.

With the formalisations presented in this thesis, we can implement, for instance, a graphical checklist to help large teams coordinate their work on complex processes in potentially safety-critical situations.
The formal basis, given an accurate specification, will ensure that no team member is left without what they need or not knowing what they have to do, and that the graphical representation is faithful to the underlying process model.
We believe this would help support people in a variety of domains, from manufacturing through administration all the way to healthcare.

It is our aim to leverage the best abilities of machines to help people perform complex tasks.
We have helped machines understand how processes are structured, and now we can make them better at assisting us.

\ifstandalone
\bibliographystyle{plainurl}
\bibliography{references}
\fi

\end{document}
