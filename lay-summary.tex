Processes are common in our daily lives and work.
The way we cook our food is a process, travelling is a process, making a train is a process.
And all these processes can be seen as collections of moment-to-moment actions that depend on each other: chopping an onion before adding it to the pan, going through security before boarding the flight, attaching the engine before placing the hood over it.

As these processes get more complex, they get more difficult to carry out.
That is made even more difficult when we are cooperating in a team.
And for some processes, such as those in healthcare or heavy manufacturing, it is vital to always carry out the process in a specific way.

In our work, we make it easier for computers to understand processes like this so that we can then better instruct them on how to assist us.
They can, for example, check that all steps have been carried out in the right order.
Or that the proposed process never gets to a point where what is needed for the next step is not present.

We do this in a highly rigorous way, mathematically proving properties of the framework we are creating so that we can trust it is doing what we expect.
In this we use a proof assistant called Isabelle/HOL, a computer program purpose-built to help us make proofs and to check that those proofs are correct.

We arrive at a system in which we can define complex processes, check for issues in their structure and visualise them.
Thanks to the proof assistant, our framework can be readily used with a number of programming languages to produce tools on a trustworthy foundation.
